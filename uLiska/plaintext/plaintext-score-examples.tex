% This file is part of the openLilyLib project
% Copyright Urs Liska 2013
% This file and the resulting PDF document aren't distributed under a free license.

\documentclass[DIV=12]{scrreprt}
\usepackage{plaintextGerman}

\usepackage{pdfpages}

\begin{document}
\begin{center}
\vspace*{3cm}
{ \Huge \textbf{\textsf{LilyPond\\
\vspace{1ex}
\huge Partitur-Beispiele}} }

\bigskip
{ \Large Urs Liska }

\emph{September 2013} 

\end{center}

\vspace{2cm}

Als Anhang zu dem Textdokument \emph{Reintext-basiertes Arbeiten für Musikwissenschaftler} zeigen die folgenden Seiten Ausschnitte exemplarischer Partituren, die mit GNU LilyPond%
\footnote{\url{http://www.lilypond.org}}
gesetzt wurden.
Die Beispiele sind in zwei Gruppen gegliedert, die jeweils eigene Aspekte darstellen sollen:
\paragraph{Veröffentlichungsqualität}
Partituren, die mit mehr oder weniger großem Aufwand im Detail perfektioniert wurden.
Diese Beispiele zeigen, dass LilyPond professionellen verlegerischen Ansprüchen genügt.
Gleichzeitig weisen diese Beispiele auch eine große stilistische Bandbreite auf.

\paragraph{Standardqualität, mit oder ohne Stilvorlagen}
Partituren, die keine nennenswerten individuellen Anpassungen aufweisen.
Diese Beispiele sind daher repräsentativ für die Qualität der vorläufigen Ergebnisse und das hohe Maß and Les- und Verwendbarkeit der Standardausgabe.
Auch wenn diese Beispiele je nach Komplexität teilweise recht weit von Publikationsqualität entfernt sind, zeigen sie, wie gut und weitgehend man diese Partituren im Arbeitsprozess verwenden kann, ohne sich um die notentypografischen Details kümmern zu müssen.
Einige Partituren entsprechen dem grundlegenden Erscheinungsbild der Software, andere wurden mit Hilfe von Stilvorlagen verändert.

Einige der Ausschnitte sind urheberrechtlich geschützten Werken entnommen und nur für den Zweck der Demonstration zusammengestellt.
Bitte respektieren Sie dies und geben das Dokument nicht an Dritte weiter.

\pagebreak
\section*{Publikationsqualität}

\begin{itemize}
\item Oskar Fried: Sommernachtslied op. 7,\,Nr.\,4
\item Martin Anton Schmid: Achtsamkeit
\item Georg Muffat: Cor vigilans
\item Mike Solomon: Aux Eppes
\item Mike Solomon: Zauberbuch
\item Julio Sagreras: Gitarrenschule
\end{itemize}

\includepdf{images/publication/fried-7-4.pdf}
\includepdf{images/publication/achtsamkeit.pdf}
\includepdf[pages=2]{images/publication/muffat.pdf}
\includepdf[pages=8,angle=90]{images/publication/auxeppes.pdf}
\includepdf[pages=21,angle=90]{images/publication/zauberbuch.pdf}
\includepdf[pages=1]{images/publication/lecon.pdf}

\section*{Standard-Ausgabe}
Partituren ohne nennenswerte typografische Korrekturen (etwa zur Kollisionsvermeidung):

\bigskip

\textbf{A) Partituren, deren Erscheinungsbild mit Hilfe von Stilvorlagen verändert wurde:}

\begin{itemize}
\item Ludwig van Beethoven: Sonate op. 10,\,3
\item Josef Suk: Bagatella 
\item Franz Schubert: Seligkeit
\item Arnold Schönberg: Erhebung op. 2,\,3
\item Julio Sagreras: El colibri
\end{itemize}

\textbf{B) Partituren mit LilyPonds Original-Erscheinungsbild:}

\begin{itemize}
\item Arnold Schönberg: Gethsemane
\item Alban Berg: Orchesterstück op. 6,\,1 (achthändige Bearbeitung)
\item Bryan Ferneyhough: La chute d'Icare
\end{itemize}

\includepdf[pages=1]{images/default/beethoven.pdf}
\includepdf[pages=1]{images/default/bagatella.pdf}
\includepdf[pages=1]{images/default/seligkeit.pdf}
\includepdf[pages=1]{images/default/erhebung.pdf}
\includepdf[pages=1]{images/default/colibri.pdf}
\includepdf[pages=1]{images/default/gethsemane.pdf}
\includepdf[pages=1]{images/default/berg.pdf}
\includepdf[pages=1]{images/default/ferneyII.pdf}

\end{document}